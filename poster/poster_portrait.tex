\documentclass[portrait,final,a0paper,fontscale=0.277]{baposter}

\usepackage{calc}
\usepackage{graphicx}
\usepackage{amsmath}
\usepackage{amssymb}
\usepackage{relsize}
\usepackage{multirow}
\usepackage{rotating}
\usepackage{bm}
\usepackage{url}

\usepackage{graphicx}
\usepackage{multicol}

%\usepackage{times}
%\usepackage{helvet}
%\usepackage{bookman}
\usepackage{palatino}

\newcommand{\captionfont}{\footnotesize}

\graphicspath{{images/}{../images/}}
\usetikzlibrary{calc}

\newcommand{\SET}[1]  {\ensuremath{\mathcal{#1}}}
\newcommand{\MAT}[1]  {\ensuremath{\boldsymbol{#1}}}
\newcommand{\VEC}[1]  {\ensuremath{\boldsymbol{#1}}}
\newcommand{\Video}{\SET{V}}
\newcommand{\video}{\VEC{f}}
\newcommand{\track}{x}
\newcommand{\Track}{\SET T}
\newcommand{\LMs}{\SET L}
\newcommand{\lm}{l}
\newcommand{\PosE}{\SET P}
\newcommand{\posE}{\VEC p}
\newcommand{\negE}{\VEC n}
\newcommand{\NegE}{\SET N}
\newcommand{\Occluded}{\SET O}
\newcommand{\occluded}{o}

%%%%%%%%%%%%%%%%%%%%%%%%%%%%%%%%%%%%%%%%%%%%%%%%%%%%%%%%%%%%%%%%%%%%%%%%%%%%%%%%
%%%% Some math symbols used in the text
%%%%%%%%%%%%%%%%%%%%%%%%%%%%%%%%%%%%%%%%%%%%%%%%%%%%%%%%%%%%%%%%%%%%%%%%%%%%%%%%

%%%%%%%%%%%%%%%%%%%%%%%%%%%%%%%%%%%%%%%%%%%%%%%%%%%%%%%%%%%%%%%%%%%%%%%%%%%%%%%%
% Multicol Settings
%%%%%%%%%%%%%%%%%%%%%%%%%%%%%%%%%%%%%%%%%%%%%%%%%%%%%%%%%%%%%%%%%%%%%%%%%%%%%%%%
\setlength{\columnsep}{1.5em}
\setlength{\columnseprule}{0mm}

%%%%%%%%%%%%%%%%%%%%%%%%%%%%%%%%%%%%%%%%%%%%%%%%%%%%%%%%%%%%%%%%%%%%%%%%%%%%%%%%
% Save space in lists. Use this after the opening of the list
%%%%%%%%%%%%%%%%%%%%%%%%%%%%%%%%%%%%%%%%%%%%%%%%%%%%%%%%%%%%%%%%%%%%%%%%%%%%%%%%
\newcommand{\compresslist}{%
\setlength{\itemsep}{1pt}%
\setlength{\parskip}{0pt}%
\setlength{\parsep}{0pt}%
}

%%%%%%%%%%%%%%%%%%%%%%%%%%%%%%%%%%%%%%%%%%%%%%%%%%%%%%%%%%%%%%%%%%%%%%%%%%%%%%
%%% Begin of Document
%%%%%%%%%%%%%%%%%%%%%%%%%%%%%%%%%%%%%%%%%%%%%%%%%%%%%%%%%%%%%%%%%%%%%%%%%%%%%%

\begin{document}

%%%%%%%%%%%%%%%%%%%%%%%%%%%%%%%%%%%%%%%%%%%%%%%%%%%%%%%%%%%%%%%%%%%%%%%%%%%%%%
%%% Here starts the poster
%%%---------------------------------------------------------------------------
%%% Format it to your taste with the options
%%%%%%%%%%%%%%%%%%%%%%%%%%%%%%%%%%%%%%%%%%%%%%%%%%%%%%%%%%%%%%%%%%%%%%%%%%%%%%
% Define some colors

%\definecolor{lightblue}{cmyk}{0.83,0.24,0,0.12}
\definecolor{lightblue}{rgb}{0.145,0.6666,1}

% Draw a video
\newlength{\FSZ}
\newcommand{\drawvideo}[3]{% [0 0.25 0.5 0.75 1 1.25 1.5]
   \noindent\pgfmathsetlength{\FSZ}{\linewidth/#2}
   \begin{tikzpicture}[outer sep=0pt,inner sep=0pt,x=\FSZ,y=\FSZ]
   \draw[color=lightblue!50!black] (0,0) node[outer sep=0pt,inner sep=0pt,text width=\linewidth,minimum height=0] (video) {\noindent#3};
   \path [fill=lightblue!50!black,line width=0pt] 
     (video.north west) rectangle ([yshift=\FSZ] video.north east) 
    \foreach \x in {1,2,...,#2} {
      {[rounded corners=0.6] ($(video.north west)+(-0.7,0.8)+(\x,0)$) rectangle +(0.4,-0.6)}
    }
;
   \path [fill=lightblue!50!black,line width=0pt] 
     ([yshift=-1\FSZ] video.south west) rectangle (video.south east) 
    \foreach \x in {1,2,...,#2} {
      {[rounded corners=0.6] ($(video.south west)+(-0.7,-0.2)+(\x,0)$) rectangle +(0.4,-0.6)}
    }
;
   \foreach \x in {1,...,#1} {
     \draw[color=lightblue!50!black] ([xshift=\x\linewidth/#1] video.north west) -- ([xshift=\x\linewidth/#1] video.south west);
   }
   \foreach \x in {0,#1} {
     \draw[color=lightblue!50!black] ([xshift=\x\linewidth/#1,yshift=1\FSZ] video.north west) -- ([xshift=\x\linewidth/#1,yshift=-1\FSZ] video.south west);
   }
   \end{tikzpicture}
}

\hyphenation{resolution occlusions}
%%
\begin{poster}%
  % Poster Options
  {
  % Show grid to help with alignment
  grid=false,
  % Column spacing
  colspacing=1em,
  % Color style
  bgColorOne=white,
  bgColorTwo=white,
  borderColor=lightblue,
  headerColorOne=black,
  headerColorTwo=lightblue,
  headerFontColor=white,
  boxColorOne=white,
  boxColorTwo=lightblue,
  % Format of textbox
  textborder=roundedleft,
  % Format of text header
  eyecatcher=true,
  headerborder=closed,
  headerheight=0.12\textheight,
%  textfont=\sc, An example of changing the text font
  headershape=roundedright,
  headershade=shadelr,
  headerfont=\Large\bf\textsc, %Sans Serif
  textfont={\setlength{\parindent}{1.5em}},
  boxshade=plain,
%  background=shade-tb,
  background=plain,
  linewidth=2pt
  }
  % Eye Catcher
  {\includegraphics[height=12.0em]{images/BAM.png}} 
  % Title
  {\bf\textsc{Test of Goodness of Population Receptive Field Estimates}\vspace{0.3em}}
  % Authors
  {\textsc{ Arash.Ashrafnejad and Hossein.Mehrzadfar and Huseyin.Boyaci @bilkent.edu.tr}\\
  \vspace{1em}
  \bf41st European Conference on Visual Perception ECVP 2018}
  % University logo
  {% The makebox allows the title to flow into the logo, this is a hack because of the L shaped logo.
    \includegraphics[height=10.0em]{images/bilkent.eps}
  }

%%%%%%%%%%%%%%%%%%%%%%%%%%%%%%%%%%%%%%%%%%%%%%%%%%%%%%%%%%%%%%%%%%%%%%%%%%%%%%
%%% Now define the boxes that make up the poster
%%%---------------------------------------------------------------------------
%%% Each box has a name and can be placed absolutely or relatively.
%%% The only inconvenience is that you can only specify a relative position 
%%% towards an already declared box. So if you have a box attached to the 
%%% bottom, one to the top and a third one which should be in between, you 
%%% have to specify the top and bottom boxes before you specify the middle 
%%% box.
%%%%%%%%%%%%%%%%%%%%%%%%%%%%%%%%%%%%%%%%%%%%%%%%%%%%%%%%%%%%%%%%%%%%%%%%%%%%%%
    %
    % A coloured circle useful as a bullet with an adjustably strong filling
    \newcommand{\colouredcircle}{%
      \tikz{\useasboundingbox (-0.2em,-0.32em) rectangle(0.2em,0.32em); \draw[draw=black,fill=lightblue,line width=0.03em] (0,0) circle(0.18em);}}

















%%%%%%%%%%%%%%%%%%%%%%%%%%%%%%%%%%%%%%%%%%%%%%%%%%%%%%%%%%%%%%%%%%%%%%%%%%%%%%
  \headerbox{Introduction}{name=introduction,column=0,row=0}{
%%%%%%%%%%%%%%%%%%%%%%%%%%%%%%%%%%%%%%%%%%%%%%%%%%%%%%%%%%%%%%%%%%%%%%%%%%%%%%
   We present a method based on computer simulations to test the goodness of population receptive field (pRF) estimates. In particular, we have examined the effect of having different hemodynamic response function (HRF) models when:
 
   \begin{enumerate}\compresslist
		\item Simulating fMRI responses using linear and non-linear "simulation HRFs"
		\item Estimating pRFs using a certain "estimation HRF"
   \end{enumerate}
   \vspace{0.3em}
 }


%%%%%%%%%%%%%%%%%%%%%%%%%%%%%%%%%%%%%%%%%%%%%%%%%%%%%%%%%%%%%%%%%%%%%%%%%%%%%%
\headerbox{Stimulation}{name=stimulus,column=1,span=2,row=0}{
%%%%%%%%%%%%%%%%%%%%%%%%%%%%%%%%%%%%%%%%%%%%%%%%%%%%%%%%%%%%%%%%%%%%%%%%%%%%%%
  \noindent We have used the drifting bar apertures at various orientations used in-[1] in order to simulate the fMRI responses.
  
    \drawvideo{24}{120}{%
        \includegraphics[width=0.0417\linewidth]{{stim_0.00_0}.png}%
        \includegraphics[width=0.0417\linewidth]{{stim_0.00_1}.png}%
        \includegraphics[width=0.0417\linewidth]{{stim_0.00_2}.png}%
        \includegraphics[width=0.0417\linewidth]{{stim_0.00_3}.png}%
        \includegraphics[width=0.0417\linewidth]{{stim_0.00_4}.png}%
        \includegraphics[width=0.0417\linewidth]{{stim_0.00_5}.png}%
        \includegraphics[width=0.0417\linewidth]{{stim_0.00_6}.png}%
        \includegraphics[width=0.0417\linewidth]{{stim_0.00_7}.png}%
        \includegraphics[width=0.0417\linewidth]{{stim_0.00_8}.png}%
        \includegraphics[width=0.0417\linewidth]{{stim_0.00_9}.png}%
        \includegraphics[width=0.0417\linewidth]{{stim_0.00_10}.png}%
        \includegraphics[width=0.0417\linewidth]{{stim_0.00_11}.png}%
        \includegraphics[width=0.0417\linewidth]{{stim_0.00_12}.png}%
        \includegraphics[width=0.0417\linewidth]{{stim_0.00_13}.png}%
        \includegraphics[width=0.0417\linewidth]{{stim_0.00_14}.png}%
        \includegraphics[width=0.0417\linewidth]{{stim_0.00_15}.png}%
        \includegraphics[width=0.0417\linewidth]{{stim_0.00_16}.png}%
        \includegraphics[width=0.0417\linewidth]{{stim_0.00_17}.png}%
        \includegraphics[width=0.0417\linewidth]{{stim_0.00_18}.png}%
        \includegraphics[width=0.0417\linewidth]{{stim_0.00_19}.png}%
        \includegraphics[width=0.0417\linewidth]{{stim_0.00_20}.png}%
        \includegraphics[width=0.0417\linewidth]{{stim_0.00_21}.png}%
        \includegraphics[width=0.0417\linewidth]{{stim_0.00_22}.png}%
        \includegraphics[width=0.0417\linewidth]{{stim_0.00_23}.png}%
    }
    \vspace{-0.6em}
}


%%%%%%%%%%%%%%%%%%%%%%%%%%%%%%%%%%%%%%%%%%%%%%%%%%%%%%%%%%%%%%%%%%%%%%%%%%%%%%
  \headerbox{References}{name=references,column=0,above=bottom}{
%%%%%%%%%%%%%%%%%%%%%%%%%%%%%%%%%%%%%%%%%%%%%%%%%%%%%%%%%%%%%%%%%%%%%%%%%%%%%%
    \smaller
    \bibliographystyle{ieee}
    \renewcommand{\section}[2]{\vskip 0.05em}
      \begin{thebibliography}{1}\itemsep=-0.01em
      \setlength{\baselineskip}{0.4em}
      \bibitem{Dumoulin2008}
        S.~Dumoulin, B.~Wandell.
        \newblock {P}opulation {R}eceptive {F}ield {E}stimates in {H}uman {V}isual {C}ortex.
        \newblock In {\em NeuroImage '08}
      \bibitem{Kay2013}
        K.~Kay and J.~Winawer et all.
        \newblock {C}ompressive {S}patial {S}ummation in {H}uman {V}isual {C}ortex.
        \newblock In {\em Journal of Neurophysiology '13}
      \end{thebibliography}
   \vspace{0.3em}
  }


%%%%%%%%%%%%%%%%%%%%%%%%%%%%%%%%%%%%%%%%%%%%%%%%%%%%%%%%%%%%%%%%%%%%%%%%%%%%%%
  \headerbox{Background Model}{name=background model,column=1,below=stimulus}{
%%%%%%%%%%%%%%%%%%%%%%%%%%%%%%%%%%%%%%%%%%%%%%%%%%%%%%%%%%%%%%%%%%%%%%%%%%%%%%
  \indent We have assumed a multivariate normal distribution for the pRF model and used log-polar varying sigma values to have a biologically plausible pRF sizes according to:
  
  \begin{equation}
  	\sigma(x, y) = \frac{1}{2} \ln(e + \sqrt{x^2 + y^2})
  \end{equation}
  which leads to eccentricity values that resemble the primary visual cortex pRF sizes.
  

  We have also used a double Gamma function as the linear HRF model which is comprised of two Gamma functions defined according to:
  
  \begin{equation}
  	content...
  \end{equation}
  
  Furthermore, we have used the Friston (1998) non-linear HRF as the non-linear
  model defined according to:
  
  \begin{equation}
  content...
  \end{equation}
  
  where, 
  
  \begin{equation}
  content...
  \end{equation}
  
   \vspace{0.3em}
  }


%%%%%%%%%%%%%%%%%%%%%%%%%%%%%%%%%%%%%%%%%%%%%%%%%%%%%%%%%%%%%%%%%%%%%%%%%%%%%%
\headerbox{Results}{name=result,column=2,row=0,below=stimulus,bottomaligned=background model}{
%%%%%%%%%%%%%%%%%%%%%%%%%%%%%%%%%%%%%%%%%%%%%%%%%%%%%%%%%%%%%%%%%%%%%%%%%%%%%%
	Accuracy maps when simulating with linear HRF and estimating with linear HRF:
   \vspace{0.3em}
  }


%%%%%%%%%%%%%%%%%%%%%%%%%%%%%%%%%%%%%%%%%%%%%%%%%%%%%%%%%%%%%%%%%%%%%%%%%%%%%%
  \headerbox{Conclusion}{name=Conclusion,column=2,below=result,above=bottom}{
%%%%%%%%%%%%%%%%%%%%%%%%%%%%%%%%%%%%%%%%%%%%%%%%%%%%%%%%%%%%%%%%%%%%%%%%%%%%%%
  \noindent
	We found that a mismatch between the HRFs may lead to erroneous pRF estimations. The errors were particularly severe when the simulation HRF was non-linear
	and the estimation HRF was linear. Therefore, we recommend that the stimulation
	protocol should be fine-tuned using computer simulations before an actual fMRI experiment is conducted.
  }


%%%%%%%%%%%%%%%%%%%%%%%%%%%%%%%%%%%%%%%%%%%%%%%%%%%%%%%%%%%%%%%%%%%%%%%%%%%%%%
  \headerbox{Source Code}{name=source,column=1,span=1,below=background model,above=bottom}{
%%%%%%%%%%%%%%%%%%%%%%%%%%%%%%%%%%%%%%%%%%%%%%%%%%%%%%%%%%%%%%%%%%%%%%%%%%%%%%
	\noindent
	\begin{minipage}{\linewidth}
		\begin{minipage}{0.7\linewidth}
			\indent{}The source code and a sample script are available at \\
		\end{minipage}\hfill%
		\begin{minipage}{0.28\linewidth}
			\hfill\includegraphics[width=\linewidth]{chart}
		\end{minipage}
	\end{minipage}
	\url{Github.com/arash-ash/pRFsim}
	}


%%%%%%%%%%%%%%%%%%%%%%%%%%%%%%%%%%%%%%%%%%%%%%%%%%%%%%%%%%%%%%%%%%%%%%%%%%%%%%
  \headerbox{Method}{name=method,column=0,below=introduction,above=references}{
%%%%%%%%%%%%%%%%%%%%%%%%%%%%%%%%%%%%%%%%%%%%%%%%%%%%%%%%%%%%%%%%%%%%%%%%%%%%%%
  A population receptive field (pRF) is the region of the visual field where stimuli illicit responses from a local population of neurons. In \cite{Dumoulin2008} the authors proposed a model of neuronal population receptive field defined by a two-dimensional Gaussian function:
  
  \begin{equation}
  g(x, y) = e^{-\frac{(x-x_0)^2+(y-y_0)^2}{2\sigma^2}}
  \end{equation}
  
  Recent study by \cite{Kay2013} showed that linear spatial summation model proposed by \cite{Dumoulin2008} deviates from linearity in early visual areas (e.g., V1, V2) and grows more in the anterior extrastriate areas (e.g., LO-2, VO-2) and the data is more accurately explained when this compressive static nonlinearity is applied after linear summation.
  
  According to the Compressive Spatial Summation (CSS) model proposed by \cite{Kay2013}, the pRF response is modeled as the element-wise (Hadamard) product of effective stimulus $s(x, y, t)$ and the Gaussian pRF model $g(x, y)$ then raised to power $n$ and multiplied with a gain factor $g$.
  \begin{equation}
  r(t) = g (\sum_{x, y} s(x, y, t) g(x, t))^n
  \end{equation}
  
  
  After generating the fMRI responses, we estimated the assumed pRF parameters by minimizing the $l2$ norm of the generated fMRI responses minus the predicted time series. Critically, we used different, as well as same HRFs for simulation and estimation.
  
   \vspace{0.3em}
  }

\end{poster}

\end{document}

