\documentclass[portrait,final,a0paper,fontscale=0.277]{baposter}

\usepackage{calc}
\usepackage{graphicx}
\usepackage{amsmath}
\usepackage{amssymb}
\usepackage{relsize}
\usepackage{multirow}
\usepackage{rotating}
\usepackage{bm}
\usepackage{url}

\usepackage{graphicx}
\usepackage{multicol}

%\usepackage{times}
%\usepackage{helvet}
%\usepackage{bookman}
\usepackage{palatino}

\newcommand{\captionfont}{\footnotesize}
\DeclareMathOperator*{\argmin}{arg\,min}
\graphicspath{{images/}{../images/}}
\usetikzlibrary{calc}

\newcommand{\SET}[1]  {\ensuremath{\mathcal{#1}}}
\newcommand{\MAT}[1]  {\ensuremath{\boldsymbol{#1}}}
\newcommand{\VEC}[1]  {\ensuremath{\boldsymbol{#1}}}
\newcommand{\Video}{\SET{V}}
\newcommand{\video}{\VEC{f}}
\newcommand{\track}{x}
\newcommand{\Track}{\SET T}
\newcommand{\LMs}{\SET L}
\newcommand{\lm}{l}
\newcommand{\PosE}{\SET P}
\newcommand{\posE}{\VEC p}
\newcommand{\negE}{\VEC n}
\newcommand{\NegE}{\SET N}
\newcommand{\Occluded}{\SET O}
\newcommand{\occluded}{o}

%%%%%%%%%%%%%%%%%%%%%%%%%%%%%%%%%%%%%%%%%%%%%%%%%%%%%%%%%%%%%%%%%%%%%%%%%%%%%%%%
%%%% Some math symbols used in the text
%%%%%%%%%%%%%%%%%%%%%%%%%%%%%%%%%%%%%%%%%%%%%%%%%%%%%%%%%%%%%%%%%%%%%%%%%%%%%%%%

%%%%%%%%%%%%%%%%%%%%%%%%%%%%%%%%%%%%%%%%%%%%%%%%%%%%%%%%%%%%%%%%%%%%%%%%%%%%%%%%
% Multicol Settings
%%%%%%%%%%%%%%%%%%%%%%%%%%%%%%%%%%%%%%%%%%%%%%%%%%%%%%%%%%%%%%%%%%%%%%%%%%%%%%%%
\setlength{\columnsep}{1.5em}
\setlength{\columnseprule}{0mm}

%%%%%%%%%%%%%%%%%%%%%%%%%%%%%%%%%%%%%%%%%%%%%%%%%%%%%%%%%%%%%%%%%%%%%%%%%%%%%%%%
% Save space in lists. Use this after the opening of the list
%%%%%%%%%%%%%%%%%%%%%%%%%%%%%%%%%%%%%%%%%%%%%%%%%%%%%%%%%%%%%%%%%%%%%%%%%%%%%%%%
\newcommand{\compresslist}{%
\setlength{\itemsep}{1pt}%
\setlength{\parskip}{0pt}%
\setlength{\parsep}{0pt}%
}

%%%%%%%%%%%%%%%%%%%%%%%%%%%%%%%%%%%%%%%%%%%%%%%%%%%%%%%%%%%%%%%%%%%%%%%%%%%%%%
%%% Begin of Document
%%%%%%%%%%%%%%%%%%%%%%%%%%%%%%%%%%%%%%%%%%%%%%%%%%%%%%%%%%%%%%%%%%%%%%%%%%%%%%

\begin{document}

%%%%%%%%%%%%%%%%%%%%%%%%%%%%%%%%%%%%%%%%%%%%%%%%%%%%%%%%%%%%%%%%%%%%%%%%%%%%%%
%%% Here starts the poster
%%%---------------------------------------------------------------------------
%%% Format it to your taste with the options
%%%%%%%%%%%%%%%%%%%%%%%%%%%%%%%%%%%%%%%%%%%%%%%%%%%%%%%%%%%%%%%%%%%%%%%%%%%%%%
% Define some colors

%\definecolor{lightblue}{cmyk}{0.83,0.24,0,0.12}
\definecolor{lightblue}{rgb}{0.145,0.6666,1}

% Draw a video
\newlength{\FSZ}
\newcommand{\drawvideo}[3]{% [0 0.25 0.5 0.75 1 1.25 1.5]
   \noindent\pgfmathsetlength{\FSZ}{\linewidth/#2}
   \begin{tikzpicture}[outer sep=0pt,inner sep=0pt,x=\FSZ,y=\FSZ]
   \draw[color=lightblue!50!black] (0,0) node[outer sep=0pt,inner sep=0pt,text width=\linewidth,minimum height=0] (video) {\noindent#3};
   \path [fill=lightblue!50!black,line width=0pt] 
     (video.north west) rectangle ([yshift=\FSZ] video.north east) 
    \foreach \x in {1,2,...,#2} {
      {[rounded corners=0.6] ($(video.north west)+(-0.7,0.8)+(\x,0)$) rectangle +(0.4,-0.6)}
    }
;
   \path [fill=lightblue!50!black,line width=0pt] 
     ([yshift=-1\FSZ] video.south west) rectangle (video.south east) 
    \foreach \x in {1,2,...,#2} {
      {[rounded corners=0.6] ($(video.south west)+(-0.7,-0.2)+(\x,0)$) rectangle +(0.4,-0.6)}
    }
;
   \foreach \x in {1,...,#1} {
     \draw[color=lightblue!50!black] ([xshift=\x\linewidth/#1] video.north west) -- ([xshift=\x\linewidth/#1] video.south west);
   }
   \foreach \x in {0,#1} {
     \draw[color=lightblue!50!black] ([xshift=\x\linewidth/#1,yshift=1\FSZ] video.north west) -- ([xshift=\x\linewidth/#1,yshift=-1\FSZ] video.south west);
   }
   \end{tikzpicture}
}

\hyphenation{resolution occlusions}
%%
\begin{poster}%
  % Poster Options
  {
  % Show grid to help with alignment
  grid=false,
  % Column spacing
  colspacing=1em,
  % Color style
  bgColorOne=white,
  bgColorTwo=white,
  borderColor=lightblue,
  headerColorOne=black,
  headerColorTwo=lightblue,
  headerFontColor=white,
  boxColorOne=white,
  boxColorTwo=lightblue,
  % Format of textbox
  textborder=roundedleft,
  % Format of text header
  eyecatcher=true,
  headerborder=closed,
  headerheight=0.12\textheight,
%  textfont=\sc, An example of changing the text font
  headershape=roundedright,
  headershade=shadelr,
  headerfont=\Large\bf\textsc, %Sans Serif
  textfont={\setlength{\parindent}{1.5em}},
  boxshade=plain,
%  background=shade-tb,
  background=plain,
  linewidth=2pt
  }
  % Eye Catcher
  {\includegraphics[height=12.0em]{images/BAM.png}} 
  % Title
  {\bf\textsc{Test of Goodness of Population Receptive Field Estimates}\vspace{0.3em}}
  % Authors
  {\textsc{ Arash.Ashrafnejad and Hossein.Mehrzadfar and Huseyin.Boyaci @bilkent.edu.tr}\\
  \vspace{1em}
  \bf41st European Conference on Visual Perception ECVP 2018}
  % University logo
  {% The makebox allows the title to flow into the logo, this is a hack because of the L shaped logo.
    \includegraphics[height=10.0em]{images/bilkent.eps}
  }

%%%%%%%%%%%%%%%%%%%%%%%%%%%%%%%%%%%%%%%%%%%%%%%%%%%%%%%%%%%%%%%%%%%%%%%%%%%%%%
%%% Now define the boxes that make up the poster
%%%---------------------------------------------------------------------------
%%% Each box has a name and can be placed absolutely or relatively.
%%% The only inconvenience is that you can only specify a relative position 
%%% towards an already declared box. So if you have a box attached to the 
%%% bottom, one to the top and a third one which should be in between, you 
%%% have to specify the top and bottom boxes before you specify the middle 
%%% box.
%%%%%%%%%%%%%%%%%%%%%%%%%%%%%%%%%%%%%%%%%%%%%%%%%%%%%%%%%%%%%%%%%%%%%%%%%%%%%%
    %
    % A coloured circle useful as a bullet with an adjustably strong filling
    \newcommand{\colouredcircle}{%
      \tikz{\useasboundingbox (-0.2em,-0.32em) rectangle(0.2em,0.32em); \draw[draw=black,fill=lightblue,line width=0.03em] (0,0) circle(0.18em);}}

















%%%%%%%%%%%%%%%%%%%%%%%%%%%%%%%%%%%%%%%%%%%%%%%%%%%%%%%%%%%%%%%%%%%%%%%%%%%%%%
  \headerbox{Introduction}{name=introduction,column=0,row=0}{
%%%%%%%%%%%%%%%%%%%%%%%%%%%%%%%%%%%%%%%%%%%%%%%%%%%%%%%%%%%%%%%%%%%%%%%%%%%%%%
	\noindent We present a method based on computer simulations to test the goodness of population receptive field (pRF) estimates. In particular, we have examined the effect of having non-linearity in the hemodynamic response function (HRF) by,
 
	\begin{enumerate}\compresslist
		\item Simulating fMRI responses with non-linear "simulation HRFs"
		\item Estimating pRFs with a "similar" linear "estimation HRF"
	\end{enumerate}
	
	This is an important scenario to test since many estimation methods assume linear HRF models but actual HRF response is non-linear. Hence we present a pipeline to test robustness of stimulation protocols against non-linearities in the BOLD response.

   \vspace{0.3em}
 }


%%%%%%%%%%%%%%%%%%%%%%%%%%%%%%%%%%%%%%%%%%%%%%%%%%%%%%%%%%%%%%%%%%%%%%%%%%%%%%
\headerbox{Stimulation}{name=stimulus,column=1,span=2,row=0}{
%%%%%%%%%%%%%%%%%%%%%%%%%%%%%%%%%%%%%%%%%%%%%%%%%%%%%%%%%%%%%%%%%%%%%%%%%%%%%%
  \noindent We first used the exact drifting bar stimululation in \cite{Dumoulin2008}:
  
    \drawvideo{24}{72}{%
        \includegraphics[width=0.0417\linewidth]{{stim_cont_0.00_0}.png}%
        \includegraphics[width=0.0417\linewidth]{{stim_cont_0.00_1}.png}%
        \includegraphics[width=0.0417\linewidth]{{stim_cont_0.00_2}.png}%
        \includegraphics[width=0.0417\linewidth]{{stim_cont_0.00_3}.png}%
        \includegraphics[width=0.0417\linewidth]{{stim_cont_0.00_4}.png}%
        \includegraphics[width=0.0417\linewidth]{{stim_cont_0.00_5}.png}%
        \includegraphics[width=0.0417\linewidth]{{stim_cont_0.00_6}.png}%
        \includegraphics[width=0.0417\linewidth]{{stim_cont_0.00_7}.png}%
        \includegraphics[width=0.0417\linewidth]{{stim_cont_0.00_8}.png}%
        \includegraphics[width=0.0417\linewidth]{{stim_cont_0.00_9}.png}%
        \includegraphics[width=0.0417\linewidth]{{stim_cont_0.00_10}.png}%
        \includegraphics[width=0.0417\linewidth]{{stim_cont_0.00_11}.png}%
        \includegraphics[width=0.0417\linewidth]{{stim_cont_0.00_12}.png}%
        \includegraphics[width=0.0417\linewidth]{{stim_cont_0.00_13}.png}%
        \includegraphics[width=0.0417\linewidth]{{stim_cont_0.00_14}.png}%
        \includegraphics[width=0.0417\linewidth]{{stim_cont_0.00_15}.png}%
        \includegraphics[width=0.0417\linewidth]{{stim_cont_0.00_16}.png}%
        \includegraphics[width=0.0417\linewidth]{{stim_cont_0.00_17}.png}%
        \includegraphics[width=0.0417\linewidth]{{stim_cont_0.00_18}.png}%
        \includegraphics[width=0.0417\linewidth]{{stim_cont_0.00_19}.png}%
        \includegraphics[width=0.0417\linewidth]{{stim_cont_0.00_20}.png}%
        \includegraphics[width=0.0417\linewidth]{{stim_cont_0.00_21}.png}%
        \includegraphics[width=0.0417\linewidth]{{stim_cont_0.00_22}.png}%
        \includegraphics[width=0.0417\linewidth]{{stim_cont_0.00_23}.png}%
    }
	\noindent Then we optimized it for angle and bar width to reduce the non-linearity:
	
    \drawvideo{24}{72}{%
	\includegraphics[width=0.0417\linewidth]{{stim_optimal_0}.png}%
	\includegraphics[width=0.0417\linewidth]{{stim_optimal_1}.png}%
	\includegraphics[width=0.0417\linewidth]{{stim_optimal_2}.png}%
	\includegraphics[width=0.0417\linewidth]{{stim_optimal_3}.png}%
	\includegraphics[width=0.0417\linewidth]{{stim_optimal_4}.png}%
	\includegraphics[width=0.0417\linewidth]{{stim_optimal_5}.png}%
	\includegraphics[width=0.0417\linewidth]{{stim_optimal_6}.png}%
	\includegraphics[width=0.0417\linewidth]{{stim_optimal_7}.png}%
	\includegraphics[width=0.0417\linewidth]{{stim_optimal_8}.png}%
	\includegraphics[width=0.0417\linewidth]{{stim_optimal_9}.png}%
	\includegraphics[width=0.0417\linewidth]{{stim_optimal_10}.png}%
	\includegraphics[width=0.0417\linewidth]{{stim_optimal_11}.png}%
	\includegraphics[width=0.0417\linewidth]{{stim_optimal_12}.png}%
	\includegraphics[width=0.0417\linewidth]{{stim_optimal_13}.png}%
	\includegraphics[width=0.0417\linewidth]{{stim_optimal_14}.png}%
	\includegraphics[width=0.0417\linewidth]{{stim_optimal_15}.png}%
	\includegraphics[width=0.0417\linewidth]{{stim_optimal_16}.png}%
	\includegraphics[width=0.0417\linewidth]{{stim_optimal_17}.png}%
	\includegraphics[width=0.0417\linewidth]{{stim_optimal_18}.png}%
	\includegraphics[width=0.0417\linewidth]{{stim_optimal_19}.png}%
	\includegraphics[width=0.0417\linewidth]{{stim_optimal_20}.png}%
	\includegraphics[width=0.0417\linewidth]{{stim_optimal_21}.png}%
	\includegraphics[width=0.0417\linewidth]{{stim_optimal_22}.png}%
	\includegraphics[width=0.0417\linewidth]{{stim_optimal_23}.png}%
}
    \vspace{-0.6em}
}


%%%%%%%%%%%%%%%%%%%%%%%%%%%%%%%%%%%%%%%%%%%%%%%%%%%%%%%%%%%%%%%%%%%%%%%%%%%%%%
  \headerbox{Experiment}{name=experiment,column=1,below=stimulus}{
%%%%%%%%%%%%%%%%%%%%%%%%%%%%%%%%%%%%%%%%%%%%%%%%%%%%%%%%%%%%%%%%%%%%%%%%%%%%%%

  \noindent We use following HRFs:
  
  \begin{itemize}
  \small
  \item Double-Gamma Linear HRF
  \begin{align*}
    p(t) &= r(t) * h(t)\\
  	h(t) &= \frac{1}{C}\frac{\lambda_1^{n_1} (t-t_1)^{n_1-1} e^{-\lambda_1(t-t_1)}}{(n_1-1)!}\\
  	&-a\frac{\lambda_2^{n_2} (t-t_2)^{n_2-1} e^{-\lambda_2(t-t_2)}}{(n_2-1)!}
  \end{align*}
  where C is the normalizing constant.
  \item Friston Non-Linear HRF
   \begin{align*}
  p(t) &= \sum_{i=1}^{3}\beta_i x_i(t)\\
   &+\sum_{i=1}^{3}\sum_{j=1}^{3}\beta_{ij} x_i(t)x_j(t)\\
   x_i(t) &= (r * b_i)(t) \\
	b_i(t) &= \frac{1}{k!} t^k e^{-t} \quad k=5,7,15
  \end{align*}
  \end{itemize}
\normalsize
	Experiment steps:
	\begin{enumerate}
		\item initialize parameters
		\begin{enumerate}
			\item number of voxels to simulate
			\item exponent of response model
			\item double-Gamma HRF
		\end{enumerate}
	\item Estimate from linear response
	\item fit parameter of non-Linear HRF
	\item Estimate from non-Linear response
	\item repeat 4 to optimize the bar width size
	\item repeat 4 with optimized bar width to optimize the rotation angle

	\end{enumerate}
  
   \vspace{0.3em}
  }


%%%%%%%%%%%%%%%%%%%%%%%%%%%%%%%%%%%%%%%%%%%%%%%%%%%%%%%%%%%%%%%%%%%%%%%%%%%%%%
\headerbox{Results}{name=result,column=2,row=0,below=stimulus,bottomaligned=experiment}{
%%%%%%%%%%%%%%%%%%%%%%%%%%%%%%%%%%%%%%%%%%%%%%%%%%%%%%%%%%%%%%%%%%%%%%%%%%%%%%
	\small
	\noindent Accuracy maps when simulating with linear HRF and estimating with linear HRF.
	\includegraphics[width=0.49\linewidth]{{0.00_y_8}.eps}
	\includegraphics[width=0.49\linewidth]{{0.00_sigma_8}.eps}
	
	\noindent Accuracy maps when simulating with non-linear HRF and estimating with linear HRF:
	\includegraphics[width=0.49\linewidth]{{cont_0.00_y_8}.eps}
	\includegraphics[width=0.49\linewidth]{{cont_0.00_sigma_8}.eps}
	
	\noindent Optimization of parameters with average of estimation errors over all pRF parameters:
	\includegraphics[width=0.45\linewidth]{barWidths}
	\includegraphics[width=0.45\linewidth]{Rotation}
	
	\noindent Accuracy maps of same experiment with optimized parameters for size and rotation:
	\includegraphics[width=0.49\linewidth]{{optimal_y_8}.eps}
	\includegraphics[width=0.49\linewidth]{{optimal_sigma_8}.eps}
   \vspace{-0.3em}
  }



%%%%%%%%%%%%%%%%%%%%%%%%%%%%%%%%%%%%%%%%%%%%%%%%%%%%%%%%%%%%%%%%%%%%%%%%%%%%%%
  \headerbox{Discussion}{name=conclusion,column=1,span=1,below=experiment,above=bottom}{
%%%%%%%%%%%%%%%%%%%%%%%%%%%%%%%%%%%%%%%%%%%%%%%%%%%%%%%%%%%%%%%%%%%%%%%%%%%%%%
	\noindent
	We found that non-linearity in simulation HRFs may lead to erroneous pRF estimations. However, we showed that it is possible to optimize the stimulution parameters to ameliorate the effect this non-linearity.\\
	
	Therefore, we highly recommend that the stimulation
	protocol (i.e., stimulation and experiment parameters) should be fine-tuned using computer simulations before an actual fMRI experiment is conducted.
	

	}




%%%%%%%%%%%%%%%%%%%%%%%%%%%%%%%%%%%%%%%%%%%%%%%%%%%%%%%%%%%%%%%%%%%%%%%%%%%%%%
  \headerbox{Method}{name=method,column=0,below=introduction, above=bottom}{
%%%%%%%%%%%%%%%%%%%%%%%%%%%%%%%%%%%%%%%%%%%%%%%%%%%%%%%%%%%%%%%%%%%%%%%%%%%%%%
  	\noindent We have developed a test based on models in \cite{Dumoulin2008}\cite{Kay2013} with following steps: \\
  	
  	
  	\noindent
    1. Parameter Initialization
      	
    Define pRF vector $\Theta = (x_0, y_0, \sigma)$.
    
     \noindent Initialize $\Theta=\theta$ such that $x_0, y_0$ are receptive field locations in the visual field corresponding to a voxel measurement and the pRF size is modeled accordingly to account for cortical mapping,
    \begin{align*}
    \sigma &= \frac{1}{2} \ln(e + \sqrt{x_0^2 + 2y_0^2})
    \end{align*} 
    
    \noindent
    2. Data Generation
  	\begin{align*}
  	g(x, y, \Theta=\theta) &= e^{-\frac{(x-x_0)^2+(y-y_0)^2}{2\sigma^2}}\\
  	r(t, \Theta=\theta) &= (\sum_{x, y} s(x, y, t) g(x,y, \theta))^n\\
  	y(t) &=  h_s(r(t)) + e(t)
	\end{align*}
	
	\noindent
	3. Parameter Estimation
  	\begin{align*}
 	r(t, \Theta) &= (\sum_{x, y} s(x, y, t) g(x, y, \Theta))^n\\
 	p(t) &= h_e(r(t, \Theta))\\
  	\hat{\Theta} &= \argmin{\sum_{t}^{}(y(t) - p(t))^2}
  	\end{align*}
  	
  	\noindent
  	4. Accuracy Map Evaluation
	\begin{align*}
  	\tilde{x} = \frac{|\hat{x_0}-x_0|}{x_0},~\tilde{y} = \frac{|\hat{y_0}-y_0|}{y_0},~\tilde{\sigma} = \frac{|\hat{\sigma}-\sigma|}{\sigma}
  	\end{align*}
  	
  	\smaller
  	\noindent with following variable descriptions:
    \begin{align*}
    x, y &= \text{point in visual space}\\
    n &= \text{spatial linearity factor}\\
  	g(x, y, \Theta) &= \text{pRF model}\\
  	s(x, y, t) &= \text{stimulation function}\\
  	r(t) &= \text{pRF response}\\
  	e(t) &= \text{Gaussian noise}\\
  	y(t) &= \text{fMRI response}\\
  	h(.) &= \text{HRF function}\\
  	\end{align*}
 
   \vspace{0.3em}
  }





%%%%%%%%%%%%%%%%%%%%%%%%%%%%%%%%%%%%%%%%%%%%%%%%%%%%%%%%%%%%%%%%%%%%%%%%%%%%%%
\headerbox{References}{name=references,column=2,below=result}{
	%%%%%%%%%%%%%%%%%%%%%%%%%%%%%%%%%%%%%%%%%%%%%%%%%%%%%%%%%%%%%%%%%%%%%%%%%%%%%%
	\smaller
	\bibliographystyle{ieee}
	\renewcommand{\section}[2]{\vskip 0.05em}
	\begin{thebibliography}{1}\itemsep=-0.01em
		\setlength{\baselineskip}{0.4em}
		\bibitem{Dumoulin2008}
		S.~Dumoulin, B.~Wandell.
		\newblock {P}opulation {R}eceptive {F}ield {E}stimates in {H}uman {V}isual {C}ortex.
		\newblock In {\em NeuroImage '08}
		\bibitem{Kay2013}
		K.~Kay and J.~Winawer et all.
		\newblock {C}ompressive {S}patial {S}ummation in {H}uman {V}isual {C}ortex.
		\newblock In {\em Journal of Neurophysiology '13}
	\end{thebibliography}
}


%%%%%%%%%%%%%%%%%%%%%%%%%%%%%%%%%%%%%%%%%%%%%%%%%%%%%%%%%%%%%%%%%%%%%%%%%%%%%%
\headerbox{Source Code}{name=source,column=2,below=references,above=bottom}{
	%%%%%%%%%%%%%%%%%%%%%%%%%%%%%%%%%%%%%%%%%%%%%%%%%%%%%%%%%%%%%%%%%%%%%%%%%%%%%%
	\smaller
	\noindent
	\begin{minipage}{\linewidth}
		\begin{minipage}{0.83\linewidth}
			\indent{} The Python \itshape{prfsim} \normalfont package can directly be installed from PyPI:\quad `pip install prfsim`


			The source code is available at, \\
				\url{github.com/arash-ash/pRFsim}
		\end{minipage}\hfill%
		\begin{minipage}{0.17\linewidth}
			\hfill\includegraphics[width=\linewidth]{chart}
		\end{minipage}
	\smaller
	\end{minipage}
	\vspace{0.3em}
}


\end{poster}

\end{document}

